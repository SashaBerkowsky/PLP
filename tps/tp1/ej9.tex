\documentclass{article}
\usepackage{graphicx} % Required for inserting images

\begin{document}
\section{Demostración}
Queremos ver que:
$$\forall t::AT\ a.\ \forall x::a.\ (elem\ x\ (preorder\ t)\ =\ elem\ x\ (postorder\ t))$$
Definiendo: 
\begin{itemize}
  \item $AT\ a = Nil\ |\ Tern\ a\ (AT\ a)\ (AT\ a)\ (AT\ a)$
  \item $preorder\ ::\ Procesador\ (AT\ a)\ a\ \\\lbrace PRE\rbrace\ preorder = foldAT\ [\ ]\ (v\ i\ c\ d\ \rightarrow\ [v]\ ++\ i\ ++\ c\ ++\ d)$
  \item $postorder\ ::\ Procesador\ (AT\ a)\ a\ \\\lbrace POST\rbrace\ postorder = foldAT\ [\ ]\ (v\ i\ c\ d\ \rightarrow\ i\ ++\ c\ ++\ [v]\ ++\ d)$
  \item $foldAT\ ::\ b\ \rightarrow\ (a\ \rightarrow\ b\ \rightarrow\ b\ \rightarrow\ b\ \rightarrow\ b)\ \rightarrow\ AT\ a\ \rightarrow\ b \\
  foldAT\ cNil\ cTern\ at = case\ at\ of \\
  \rule{1cm}{0pt}Nil\ \rightarrow\ cNil \\
  \rule{1cm}{0pt}Tern\ v\ c\ i\ d\ \rightarrow\ cTern\ val\ (rec\ izq)\ (rec\ cen)\ (rec\ der) \\
  \rule{1cm}{0pt}where\ rec=\ foldAT cNil cTern\\$
\end{itemize}
Por inducción estructural en $t$ definimos el enunciado $P(t)$ y planteamos tanto el caso base como el caso inductivo:\\\\
$\forall i,\ c,\ d::AT\ a.\ \forall v::a\\\\$
$P(t)=elem\ x\ (preorder\ t)=elem\ x\ (postorder\ t)$
\begin{itemize}
    \item Caso base: $P(Nil)$
    \item Caso inductivo: $(P(i)\ \land\ P(c)\ \land\ P(d))\Rightarrow P(AT\ v\ i\ c\ d)$
\end{itemize}
\subsection{Caso Base}
$P(Nil)\equiv elem\ x\ (preorder\ Nil)=elem\ x\ (postorder\ Nil)\\\\$
Reemplazando en preorder/postorder y luego en foldAT:\\
$preorder\ Nil = [\ ]\\$
$postorder\ Nil = [\ ]\\\\$
$P(Nil)\equiv elem\ x\ [\ ]=elem\ x\ [\ ]\\$
$elem\ x\ [\ ]=False \Rightarrow P(Nil) \equiv False = False\\\\$
$P(Nil)=True$
\subsection{Caso Inductivo}
$(P(i)\ \land\ P(c)\ \land\ P(d))\Rightarrow P(AT\ v\ i\ c\ d)$\\\\
Para demostrar esta implicación asumimos la implicación\\\\
Con este criterio vemos que $P(i)\ \land\ P(c)\ \land P(d)$ es verdadero, con lo cual:
\begin{itemize}
    \item $P(i)\equiv elem\ x\ (preorder\ i) =elem\ x\ (postorder\ i) \equiv True$
    \item $P(c)\equiv elem\ x\ (preorder\ c) =elem\ x\ (postorder\ c) \equiv True$
    \item $P(d)\equiv elem\ x\ (preorder\ d) =elem\ x\ (postorder\ d) \equiv True$
\end{itemize}\\
Pasamos a detallar el consecuente:\\\\
$P(AT\ v\ i\ c\ d)\equiv elem\ x\ (preorder\ (AT\ v\ i\ c\ d))=elem\ x\ (preorder\ (AT\ v\ i\ c\ d))\\\\$
Reemplazando en preorder/postorder y luego en foldAT:\\
$preorder\ (AT\ v\ i\ c\ d)=[v]\ ++\ preorder\ i\ ++\ preorder\ c\ ++\ preorder\ d\\$
$postorder\ (AT\ v\ i\ c\ d)=postorder\ i\ ++\ postorder\ c\ ++\ [v]\ ++\ postorder\ d\\\\$
Por lo tanto:\\
$P(AT\ v\ i\ c\ d)\equiv\\ elem\ x\ ([v]\ ++\ preorder\ i\ ++\ preorder\ c\ ++\ preorder\ d)\\=elem\ x\ (postorder\ i\ ++\ postorder\ c\ ++\ [v]\ ++\ postorder\ d)\\\\$
Ahora desarrollamos cada lado de $P(AT\ v\ i\ c\ d)$:
\subsubsection{Lado Izquierdo}
$elem\ x\ ([v]\ ++\ preorder\ i\ ++\ preorder\ c\ ++\ preorder\ d)=\\$
$x=v\ \lor\ elem\ x\ (preorder\ i)\ \lor\ elem\ x\ (preorder\ c)\ \lor\ elem\ x\ (preorder\ d)$
\subsubsection{Lado Derecho}
$elem\ x\ (postorder\ i\ ++\ postorder\ c\ ++\ [v] ++\ postorder\ d)=$
\[x=v\ \lor\ elem\ x\ (postorder\ i)\ \lor\ elem\ x\ (postorder\ c)\ \lor\ elem\ x\ (postorder\ d)\]
\\ Así podemos separar estos 2 casos:
\begin{enumerate}
  \item $x = v$
  \item $x\neq v$
\end{enumerate}
\subsubsection{$x = v$}
$x=v\ \lor\ elem\ x\ (preorder\ i)\ \lor\ elem\ x\ (preorder\ c)\ \lor\ elem\ x\ (preorder\ d)\equiv elem\ x\ (preorder\ AT\ v\ i\ c\ d)\equiv True$\\\\
$x=v\ \lor\ elem\ x\ (postorder\ i)\ \lor\ elem\ x\ (postorder\ c)\ \lor\ elem\ x\ (postorder\ d)\equiv elem\ x\ (postorder\ AT\ v\ i\ c\ d)\equiv True$\\\\
$\Rightarrow elem\ x\ (preorder AT\ v\ i\ c\ d)=elem\ x\ (postorder\ AT\ v\ i\ c\ d)\equiv True$\\
$\Rightarrow P(AT\ v\ i\ c\ d)\equiv True$
\subsubsection{$x \neq v$}
Como $x\neq v$ podemos detallar ambos lados de $P(AT\ v\ i\ c\ d)$ de la siguiente manera:$\\\\$
Lado izquierdo: $elem\ x\ (preorder\ i)\ \lor\ elem\ x\ (preorder\ c)\ \lor\ elem\ x\ (preorder\ d)\\$
Lado derecho: $elem\ x\ (postorder\ i)\ \lor\ elem\ x\ (postorder\ c)\ \lor\ elem\ x\ (postorder\ d)\\\\$
Ahora bien, por hipótesis inductiva sabememos que
\begin{itemize}
    \item $P(i)\equiv elem\ x\ (preorder\ i) =elem\ x\ (postorder\ i) \equiv True$
    \item $P(c)\equiv elem\ x\ (preorder\ c) =elem\ x\ (postorder\ c) \equiv True$
    \item $P(d)\equiv elem\ x\ (preorder\ d) =elem\ x\ (postorder\ d) \equiv True$
\end{itemize}
Con lo cual
$\\\\elem\ x\ (preorder\ i)\ \lor\ elem\ x\ (preorder\ c)\ \lor\ elem\ x\ (preorder\ d)=\\$
$elem\ x\ (postorder\ i)\ \lor\ elem\ x\ (postorder\ c)\ \lor\ elem\ x\ (postorder\ d)\\
\equiv P(AT\ v\ i\ c\ d)\equiv True\\\\$
De esta forma demostramos por inducción estructural que
$$\forall t::AT\ a.\ \forall x::a.\ (elem\ x\ (preorder\ t)\ =\ elem\ x\ (postorder\ t))$$
\end{document}
